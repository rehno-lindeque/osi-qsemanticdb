%%%%%%%%%%%%%%%%%%%%%%%%%%%%%%%%%%%%%%%%%%%%%%%%%%%%%%%%%%%%%%%%%%%%%%%%%%%%%%
%
%     ARCHITECTURE.TEX
%
%     Copyright © 2009, Rehno Lindeque. All rights reserved.
%
%%%%%%%%%%%%%%%%%%%%%%%%%%%%%%%%%%%%%%%%%%%%%%%%%%%%%%%%%%%%%%%%%%%%%%%%%%%%%%

\documentclass[a4paper,11pt]{article}

%                                  PACKAGES                                  %
%\usepackage{tikz}
\usepackage{geometry}
\geometry{a4paper}
%\usepackage{float}
%\usepackage{wrapfig}
%\usepackage{proof}

%\usepackage{amssymb}
%\usepackage{amsthm}
%\usepackage{amsmath}

%                                   MACROS                                   %

%                                  DOCUMENT                                  %
\begin{document}

\title{Architecture of the QSemanticDB implementation}
\author{Rehno Lindeque}

\maketitle

\begin{abstract}
The architecture an OpenSemanticDB reference implementation is described.
\end{abstract}

\section{Introduction}

\subsection{Terminology}
\begin{itemize}
\item \emph{Semantic symbol / symbol} -- The conceptual representation of a semantic value. For example $a$ could be a symbol and so could $a.b$
\item \emph{Semantic id / id} -- The internal representation of some semantic symbol.
\item \emph{Fully qualified id} -- An id that represents some semantic symbol inside some context.
\item \emph{Unqualified id} -- An id that represents some semantic symbol disregarding the context it is found in. (I.e. $a.(b.c)$ is qualified, but $c$ or even $b.c$ is unqualified when refering to a).
\item \emph{Relation} -- A declaration in the form $a \rightarrow b$ which is represented by the existence of the query $a.b$.
\item \emph{Unqualified relation} -- A pair of id's $(domainId, codomainId)$ such that $codomainId$ is unqualified with respect to $domainId$.
\end{itemize}

\subsection{Data structures}
The following basic indexes are needed. Note that a $qualifiedCodomain$ represents the relation $domain.unqualifiedCodomain$.
\begin{itemize}
\item SplitRelations: $qualifiedCodomain \rightarrow (domain, unqualifiedCodomain)$
\item UnifyRelations: $(domain, unqualifiedCodomain) \rightarrow qualifiedCodomain$
\item DomainQCodomains $domain \rightarrow \* qualifiedCodomain$
\item DomainUCodomains $domain \rightarrow \* unqualifiedCodomain$
\end{itemize}

\end{document}
